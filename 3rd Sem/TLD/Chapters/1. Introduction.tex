\section{Objective}
\begin{itemize}
    \item  To study the characteristics of CaSO4: Dy.
    \item To calibrate the TL/OSL research Reader in terms of absorbed dose and find out the unknown dose from a sample.
\end{itemize}
\section{Apparatus}
\begin{itemize}
    \item Annealed TLD samples (CaSO4: Dy)
    \item Slab phantoms
    \item TL/OSL Research reader
    \item TLD Annealing Oven
    \item Radiation-generating equipment
\end{itemize}
\section{Theory}
\section{Observation}
\begin{table}[H]
\centering
\caption{Intensity values for different doses}
\begin{tabular}{ccccccccc}
\hline
\textbf{Dose (Gy)} & \textbf{A} & \textbf{B} & \textbf{C} & \textbf{D} & \textbf{E} & \textbf{Net Intensity} \\
\hline
Background & 842326 & 8401365 & 7377735 & 1443072 & 12047312 & 0 \\
0.5 & 42366729 & 46893479 & 36180595 & 48441922 & 54392237 & 39632630.4 \\
1 & 72423155 & 80360004 & 73245158 & 82007274 & 73189478 & 70222651.8 \\
2 & 94610261 & 93586371 & 99646954 & 94782233 & 91800276  & 88862857 \\
3 & 97315644 & 10255814 & 110922831 & 111687756 & 108736283  & 100221763.6 \\
4 & 118198629 & 117279687 & 121895156 & 112600457 & 117156973  & 111403818.4 \\
6 & 129900901 & 129918209 & 134048179 & 127953661 & 126380141  & 123474018.2 \\
Unknown A & 105077687 & 107876926 & 107610287 & 99717134 & 104853017  & 99004648.2 \\
Unknown B & 119925678 & 120381880 & 116170288 & 118524838 & 120983219  & 113174818.6 \\
\hline
\end{tabular}
\end{table}


\begin{figure}[H]
    \centering
    \includegraphics[width=0.7\textwidth]{/Users/souvikpc/Desktop/Lab-Reports/3rd Sem/TLD/Figures/Logos/Experimental data/TL intensity vs Dose.pdf}
    \caption{Dose vs Average Intensity graph for CaSO4: Dy TLD.}
\end{figure}
From the graph, we can see that the relationship between dose and intensity is linear. Using the linear fit equation, we can calculate the unknown doses.
Using the linear fit equation:
\[\boxed{\text{Dose} = \frac{I - 5.12\times10^7}{1.37\times10^{7}}}
\]
Where I = Intensity. So for unknown A: the intensity is 99004648.2
\[\text{Dose} = \frac{99004648.2 - 5.12\times10^7}{1.37\times10^{7}} = \boxed{3.489} \text{ Gy}\]
Similarly, for unknown B: the intensity is 113174818.6
\[\text{Dose} = \frac{113174818.6 - 5.12\times10^7}{1.37\times10^{7}} = \boxed{4.524} \text{ Gy}\]
So the relative errors in the calculated doses are:
For unknown A:



\subsection{Heating rate analysis}
\begin{table}[H]
\centering
\caption{Intensity values for different doses}
\begin{tabular}{ccccccc}
\hline
\textbf{Heating rate(°C/s)} & \textbf{A} & \textbf{B} &\textbf{C} & \textbf{Bg A} & \textbf{Bg B} & \textbf{Net Intensity} \\
\hline
1 & 128731197 & 156087955 & 130975226 & 4211945 & 28177825 & 122403241 \\
1.5 & 91896493 & 106535449 & 111177232 & 5737988 & 34929935 & 82869096.5 \\
2 & 97266471 & 77432993 & 109717659 & 17155082 & 2693598 & 84881367.67 \\
2.5 & 93535507 & 81280319 & 87006715 & 15384092 & 23027192 & 68068538.33 \\
3 & 88523949 & 75226109 & 72113909 & 7197779 & 16678551 & 66683157.33 \\
4 & 69037517 & 63641046 & 65486040 & 69037517 & 307269 & 31382474.67 \\
\hline
\end{tabular}
\end{table}

\begin{figure}[H]
    \centering
    \includegraphics[width=0.7\textwidth]{/Users/souvikpc/Desktop/Lab-Reports/3rd Sem/TLD/Figures/Logos/Experimental data/Heatingrate.pdf}
    \caption{ Average Intensity graph for CaSO4: Dy TLD vs Heating rate}
\end{figure}
\section{Conclusion}