\section{Objective}
\begin{itemize}
    \item  To study the characteristics of CaSO4: Dy.
    \item To calibrate the TL/OSL research Reader in terms of absorbed dose and find out the unknown dose from a sample.
\end{itemize}
\section{Apparatus}
\begin{itemize}
    \item Annealed TLD samples (CaSO4: Dy)
    \item Slab phantoms
    \item TL/OSL Research reader
    \item TLD Annealing Oven
    \item Radiation-generating equipment
\end{itemize}
\section{Theory}
In a perfect crystalline insulator, the conduction and valence bands are separated by an energy
difference of several eV, and there are no intermediate energy levels within the band gap. 
Lumininescence detectors are created by adding impurities (activators) to the crystal structure that
create energy levels within the band gap. When ionizing radiation interacts with the crystal, it
excites electrons from the valence band to the conduction band(Fig. \ref{fig:mechanism}a), leaving behind holes in the valence
band. Some of these electrons and holes become trapped at the impurity levels, creating metastable
states. The trapped electrons and holes can later be released by providing energy, such as heat
(thermoluminescence) or light (optically stimulated luminescence). In the absence of external
stimulation, the trapped charges can remain in their metastable states for extended periods,
ranging from hours to years, depending on the stability of the traps.

During the readout process [Fig. \ref{fig:mechanism}(b)], stimulation by heat
(TL) or light (OSL) releases the trapped charges. Once the
trapped electron is released, it can recombine with the trapped
hole, creating a defect in the excited state. TL or OSL results
from the relaxation of these defects to return to the ground
state by light emission. One should keep in mind that this
model is simplified. Actual TL and OSL materials have many
defects that may not give rise to observable TL/OSL signals
but may act as competitors for the transitions represented in
Fig. \ref{fig:mechanism}, causing changes in the TL/OSL properties. These
competitors give rise to phenomena such as supralinearity of
the dose response and sensitivity changes with detector's dose
and annealing history.


\begin{figure}[H]
    \centering
    \includegraphics[width=0.7\textwidth]{/Users/souvikpc/Desktop/Lab-Reports/3rd Sem/TLD/Figures/Theory/Mechanism.png}
    \caption{Simplified energy level diagram representing the delocalized bands (valence band and conduction band) and the electronic transitions in Thermolumines-
cent/optically stimulated luminescent material during irradiation (a) and readout (b) where stimulation is provided by heat or light. In this illustration, the hole
trap is more stable than the electron trap. Therefore, the electron trap is said to act as a trapping center, whereas the hole trap is said to act as a recombination
center.\cite{https://doi.org/10.1002/mp.13839}}
    \label{fig:mechanism}
\end{figure}

Thermoluminescent dosimeters are read by increasing the
temperature of the detector while monitoring the TL emitted,
either creating a glow curve graph of signal vs temperature or
by recording the total signal for a portion of the heating pro-
cess. In the former, the TL signal can be defined as the maxi-
mum intensity of a TL peak or the integrated TL intensity
over a region of interest. Figure \ref{fig:tl100} illustrates a glow curve
for LiF:Mg,Ti (TLD-100). Different trapping centers have
different depths within the bandgap that require different tem-
peratures for release. As the detector is heated, these trapping
centers are stimulated sequentially, giving rise to a series of
peaks. 

\begin{figure}[H]
    \centering
    \includegraphics[width=0.5\textwidth]{/Users/souvikpc/Desktop/Lab-Reports/3rd Sem/TLD/Figures/Theory/TL100.png}
    \caption{Thermoluminescent curves of LiF:Mg,Ti (TLD-100) subjected to different preirradiation annealing treatments. Peaks 1–5 correspond to different trapping centers (which yield signal at different temperatures); peak 5 occurs at the temperature of maximum signal.\cite{https://doi.org/10.1002/mp.13839}}
    \label{fig:tl100}
\end{figure}

Both the temperature of the maximum peak and the
maximum TL signal depend on the heating rate. To minimize
the effect of heating rate fluctuations, the integral of the TL
signal rather than its maximum is typically used to determine
the dose. TL peaks with maxima around 200-225°C are best
for dosimetry because of their stability at room temperature
(Peak 5 in Fig. \ref{fig:tl100}). TL peaks at lower temperatures are due to
shallow trapping levels in the band gap and may be unstable
at room temperature. TL peaks appearing at higher temperatures may suffer interference from infrared black-body 
radiation and spurious signals. These effects are mitigated by
using a bandpass filter to eliminate the infrared signals and
by flowing an inert gas such as $N_2$ or $Ar$ over the TL 
phosphor during heating to reduce spurious signals. The relative
importance of these peaks and the sensitivity of the detector
will depend on annealing regimens before irradiation.
Because the TL readout depletes the majority of the trapping
centers, less than 1\% of the original signal should be
observed if the material is read again without irradiation.
\begin{figure}[H]
    \centering
    \includegraphics[width=0.5\textwidth]{/Users/souvikpc/Desktop/Lab-Reports/3rd Sem/TLD/Figures/Theory/TLD reader.png}
    \caption{Schematic diagram of a typical thermoluminescent dosimeter (TLD) reader.\cite{https://doi.org/10.1002/mp.13839}}
    \label{fig:tld reader}
\end{figure}
While different peaks can include different information
about the irradiation, in clinical practice the standard practice
is to simply integrate the entire glow curve to yield the overall
signal. To ensure reproducible results, a consistent heating
cycle should be used.

Figure \ref{fig:tld reader} presents a typical configuration for a TLD
reader. The heating source can be an ohmically heated pan,
preheated $N_2$ gas, or an infrared light source. A typical heating cycle consists of a rapid increase to a moderate
temperature during which the signal, primarily from unstable
traps, is not recorded, followed by a linear temperature
increase during which the TL glow curve is measured (e.g.,
peak 5 in Fig. \ref{fig:tl100}). The final step is heating the crystal to a
high temperature to anneal the TLD, emptying the deep traps
that were untouched at lower temperatures.


\subsection{Types of TLD/OSLD}

Thermoluminescent dosimeter (TLD) and optically stimulated luminescent dosimeter (OSLD) materials available commercially along with example commercial names, density $\rho$, effective atomic number Zeff, temperature of the main TL peak, and typical emission wavelength, sensitivity relative to LiF and fading.
\begin{table}[ht]
\centering
\resizebox{\textwidth}{!}{
\begin{tabular}{l l c c c c c c c}
\hline
Material & Commercial Name & $\rho$ & $Z_{\text{eff}}$ & Glow Peak ($^\circ$C) & Emission (nm) & TL Sens. & Fading \\
\hline

LiF:Mg,Ti & TLD-100 & 2.6 & 8.31 & $\sim$235 & $\sim$410 & Referent & 5\% in 3--12 months \\

$^{6}$LiF:Mg,Ti & TLD-600 & 2.6 & 8.31 & $\sim$235 & $\sim$410 & 1.0 & 5\% in 3--12 months \\

$^{7}$LiF:Mg,Ti & TLD-700 & 2.6 & 8.31 & $\sim$235 & $\sim$410 & 1.0 & 5\% in 3--12 months \\

LiF:Mg,Cu,P & TLD-100H & 2.5 & 8.31 & $\sim$200 & $\sim$370 & 30  & 2\% in 3 months \\

Li$_2$B$_4$O$_7$:Mn & TLD-800 & 2.3 & 7.32 & $\sim$185 & $\sim$600 & 0.3 & 5--10\% in 3 months \\

CaF$_2$:Dy & TLD-200 & 3.18 & 16.90 & $\sim$160,185,245,290 & 480,575,660,750 & 30 & 25\% in 4 weeks \\

CaF$_2$:Mn & TLD-400 & 3.18 & 16.90 & $\sim$300 & $\sim$495 & 10 & 15\% in 2--4 weeks \\

\textbf{CaSO$_4$:Dy} & \textbf{TLD-900} & \textbf{2.61} & \textbf{15.62} & $\sim$220 & \textbf{480,575,660,750} & \textbf{15} & \textbf{6\% in 6 months} \\

\hline
\end{tabular}
}
\end{table}

\subsection{TLD forms}
\begin{figure}[H]
    \centering
    \includegraphics[width=0.7\textwidth]{/Users/souvikpc/Desktop/Lab-Reports/3rd Sem/TLD/Figures/Theory/TLD forms.png}
    \caption{Different forms of TLDs \cite{https://doi.org/10.1002/mp.13839}}
    \label{fig:tld forms}
\end{figure}

Thermoluminescent dosimeters can be purchased in different forms that offer different advantages and disadvantages. 
The main forms are powder and solid (Fig. \ref{fig:tld forms}), including rods, chips, disks, and microcubes. With powder it is possible
 to generate a large number of TLD with uniform characteristics. However, powder is challenging to use in terms of loading,
  handling, and annealing. Powder form is also sensitive to mass and powder distribution during readout. The mass of the powder
   is important: if the mass is too large, excessive self-attenuation of the TL occurs, which decreases the apparent sensitivity.
    Problems also arise for small masses. For masses $<$ 18 mg, the signal/mass becomes dependent on the mass; therefore,
     only masses $>$ 20 mg should be used. The distribution of powder on the heating pan also requires attention, 
     as the centering and distribution of the powder can affect the signal/mass by several percent. In contrast
      to a powder form, solid forms can readily be annealed and reused indefinitely if appropriate procedures are 
      followed. However, because each detector is unique, each will show variability in sensitivity that needs to
       be accounted for. Solid forms are also sensitive to being scratched or chipped, and additional care is required in their handling.



\section{Observation}
The TLDs underwent irradiation at specified doses employing a medical linear accelerator. These
TLD specimens were positioned within a solid water phantom that featured pre-engineered grooves to
securely maintain the detectors alignment. For sample irradiation, a Source-to-Surface Distance (SSD)
arrangement was utilized. A 2 cm build-up area was established above the TLDs to maintain charged
particle equilibrium, with additional slabs positioned beneath the samples to manage backscatter
radiation. This irradiation process was conducted using a Versa HD linear accelerator equipped with
a 6 MV photon beam.

\begin{table}[H]
\centering
\caption{Intensity values for different doses}
\begin{tabular}{ccccccccc}
\hline
\textbf{Dose (Gy)} & \textbf{A} & \textbf{B} & \textbf{C} & \textbf{D} & \textbf{E} & \textbf{Net Intensity} \\
\hline
Background & 842326 & 8401365 & 7377735 & 1443072 & 12047312 & 0 \\
0.5 & 42366729 & 46893479 & 36180595 & 48441922 & 54392237 & 39632630.4 \\
1 & 72423155 & 80360004 & 73245158 & 82007274 & 73189478 & 70222651.8 \\
2 & 94610261 & 93586371 & 99646954 & 94782233 & 91800276  & 88862857 \\
3 & 97315644 & 10255814 & 110922831 & 111687756 & 108736283  & 100221763.6 \\
4 & 118198629 & 117279687 & 121895156 & 112600457 & 117156973  & 111403818.4 \\
6 & 129900901 & 129918209 & 134048179 & 127953661 & 126380141  & 123474018.2 \\
Unknown A & 105077687 & 107876926 & 107610287 & 99717134 & 104853017  & 99004648.2 \\
Unknown B & 119925678 & 120381880 & 116170288 & 118524838 & 120983219  & 113174818.6 \\
\hline
\end{tabular}
\end{table}


\begin{figure}[H]
    \centering
    \includegraphics[width=0.7\textwidth]{/Users/souvikpc/Desktop/Lab-Reports/3rd Sem/TLD/Figures/Logos/Experimental data/TL intensity vs Dose.pdf}
    \caption{Dose vs Average Intensity graph for CaSO4: Dy TLD.}
\end{figure}
From the graph, we can see that the relationship between dose and intensity is linear. Using the linear fit equation, we can calculate the unknown doses.
Using the linear fit equation:
\[\boxed{\text{Dose} = \frac{I - 5.12\times10^7}{1.37\times10^{7}}}
\]
Where I = Intensity. So for unknown A: the intensity is 99004648.2
\[\text{Dose} = \frac{99004648.2 - 5.12\times10^7}{1.37\times10^{7}} = \boxed{3.489} \text{ Gy}\]
Similarly, for unknown B: the intensity is 113174818.6
\[\text{Dose} = \frac{113174818.6 - 5.12\times10^7}{1.37\times10^{7}} = \boxed{4.524} \text{ Gy}\]
So the relative errors in the calculated doses are:
\subsubsection*{Error analysis}
\begin{table}[H]
\centering
\begin{tabular}{cccc}
\hline
\textbf{Unknown} & \textbf{Calculated Dose (Gy)} & \textbf{Actual dose (Gy)}& \textbf{Relative Error (\%)} \\
\hline
A & 3.49 & 3.35 & 4.18 \% \\
B & 4.52 & 4.47 &  1.12 \% \\
\hline
\end{tabular}
\end{table}



\subsection{Heating rate analysis}
Thermoluminescent dosimeter (TLD) samples administered with an identical dose are analyzed with
respect to the heating rate: 1 °C/s, 1.5 °C/s, 2 °C/s, 2.5 °C/s, 3 °C/s and 4 °C/s.

\begin{figure}[H]
    \centering
    \includegraphics[width=0.7\textwidth]{/Users/souvikpc/Desktop/Lab-Reports/3rd Sem/TLD/Figures/Logos/Experimental data/Intensity vs Temparature for defferent heating rate.pdf}
    \caption{Glow Curve for different heating rates}
\end{figure}

From this graph, we can see that as the heating rate increases, the peak of the glow curve shifts to higher temperatures and the intensity decreases. 
Because at higher heating rates, the trapped electrons have less time to be released and recombine with holes, leading to a lower intensity of emitted light.

\begin{table}[H]
\centering
\caption{Intensity values for different doses}
\begin{tabular}{ccccccc}
\hline
\textbf{Heating rate(°C/s)} & \textbf{A} & \textbf{B} &\textbf{C} & \textbf{Bg A} & \textbf{Bg B} & \textbf{Net Intensity} \\
\hline
1 & 128731197 & 156087955 & 130975226 & 4211945 & 28177825 & 122403241 \\
1.5 & 91896493 & 106535449 & 111177232 & 5737988 & 34929935 & 82869096.5 \\
2 & 97266471 & 77432993 & 109717659 & 17155082 & 2693598 & 84881367.67 \\
2.5 & 93535507 & 81280319 & 87006715 & 15384092 & 23027192 & 68068538.33 \\
3 & 88523949 & 75226109 & 72113909 & 7197779 & 16678551 & 66683157.33 \\
4 & 69037517 & 63641046 & 65486040 & 69037517 & 307269 & 31382474.67 \\ \hline
\end{tabular}
\end{table}

\begin{figure}[H]
    \centering
    \includegraphics[width=0.7\textwidth]{/Users/souvikpc/Desktop/Lab-Reports/3rd Sem/TLD/Figures/Logos/Experimental data/Heatingrate.pdf}
    \caption{ Average Intensity graph for CaSO4: Dy TLD vs Heating rate}
\end{figure}

From the graph, we can see that the relationship between heating rate and intensity is decreasing exponentially.


\section{Applications of TLDs}
\begin{enumerate}[leftmargin=*,label=\arabic*.]
  \item \textbf{Patient dosimetry}
    \begin{itemize}
      \item In-vivo surface dose verification during radiotherapy.
      \item Measurement of entrance/exit doses and dose to critical points.
      \item Verification for special setups (bolus, immobilization devices, etc.).
    \end{itemize}

  \item \textbf{Calibration and beam quality checks}
    \begin{itemize}
      \item Independent dose verification of linear accelerators and other treatment units.
      \item Participation in external postal audits and inter-laboratory comparisons.
    \end{itemize}

  \item \textbf{Brachytherapy dosimetry}
    \begin{itemize}
      \item Point-dose measurements near LDR/HDR sources (e.g., I-125, Ir-192).
      \item Mapping dose gradients around brachytherapy implants and applicators.
    \end{itemize}

  \item \textbf{Diagnostic radiology}
    \begin{itemize}
      \item Low-dose measurements in CT, mammography, fluoroscopy and radiography.
      \item Organ or phantom dose estimation for protocol optimization and QA.
    \end{itemize}

  \item \textbf{Radiation protection / Occupational monitoring}
    \begin{itemize}
      \item Personal dosimetry (staff badges) in hospitals, nuclear facilities, and research labs.
      \item Area monitoring for controlled or supervised radiation zones.
    \end{itemize}

  \item \textbf{Research and development}
    \begin{itemize}
      \item Radiation transport experiments, detector comparisons, and material studies.
      \item Use in controlled experiments where precise point-dose readings are needed.
    \end{itemize}

  \item \textbf{Space and aviation dosimetry}
    \begin{itemize}
      \item Monitoring cosmic radiation exposure for astronauts and high-altitude crew.
      \item Long-term passive monitoring in satellites and space hardware.
    \end{itemize}

  \item \textbf{Small-field and high-gradient dosimetry}
    \begin{itemize}
      \item Dose measurement for stereotactic radiosurgery (SRS) and stereotactic body radiotherapy (SBRT).
      \item Measurements in regions with steep dose gradients or near tissue--air interfaces.
    \end{itemize}
\end{enumerate}

\section{Conclusion}
In this experiment, we explored the principles and applications of Thermoluminescence Dosimetry (TLD) using CaSO4: Dy as the dosimetric material. The linear relationship between dose and intensity was established, allowing for accurate dose measurements. The analysis of heating rates revealed their impact on glow curve characteristics, emphasizing the importance of controlled readout conditions. The calculated doses for unknown samples demonstrated the reliability of TLD in clinical dosimetry, with relative errors within acceptable limits. Overall, this experiment highlighted the effectiveness of TLD in radiation dose measurement and its potential applications in medical physics.