\section{Objective}
\begin{itemize}
    \item Energy calibration of HPGe detector
    \item Absolute efficiency calibration of HPGe detector
    \item Identification of radionuclides and measurement of their activity 
    concentration present in environmental soil sample and tap water sample.
\end{itemize}
\section{Apparatus}
\subsection{Introduction}
High Purity Germanium (HPGe) detectors are semiconductor detectors widely 
used in gamma spectroscopy due to their superior energy resolution compared to 
other detector types like NaI(Tl) scintillation detectors. They are particularly 
effective for identifying and quantifying gamma-emitting radionuclides in various
samples, including environmental samples.\vspace{2 mm}

Due to higher atomic number (Z=32), Ge has a much larger linear attenuation
coefficient for gamma rays than Si (Z=14). Which leads to a shorter mean free path
for gamma rays in Ge than in Si. This property makes Ge detectors more efficient
at detecting gamma rays, especially at higher energies.\vspace{2 mm}

The HPGe detectors must be operate at cryogenic temperatures (typically around 77 K,
the boiling point of liquid nitrogen) to reduce thermal noise and improve their
performance. This cooling is necessary because germanium has a relatively small bandgap,
and at room temperature, thermal energy can excite electrons across the bandgap,
leading to increased noise and reduced energy resolution.

\subsection{Principle of operation}
When a gamma ray interacts with the germanium crystal in the HPGe detector,
it can undergo several types of interactions, including the photoelectric effect,
Compton scattering, and pair production. These interactions result in the creation
of electron-hole pairs within the crystal. \textbf{The number of electron-hole pairs generated
is proportional to the energy of the incoming gamma ray}\vspace{2 mm}

An electric field is applied across the germanium crystal, causing the electrons
and holes to drift towards their respective electrodes. This movement of charge carriers
generates a current pulse, which is then amplified and processed by the detector's electronics.
The amplitude of the pulse is directly proportional to the energy of the incident gamma ray,
allowing for precise energy measurements.\vspace{2 mm}

\subsection{Construction and Cryostatic colling}
The HPGe detector consists of a high-purity germanium crystal, typically in the form
of a cylindrical or coaxial shape. The crystal is housed in a vacuum-sealed cryostat
to maintain the low temperatures required for operation. The cryostat is often cooled
using liquid nitrogen, which is circulated around the detector to keep it at the
desired temperature.\vspace{2 mm}

The main components of the system are the cooling rod (usually made of copper), the Dewar
container (for storage of liquid nitrogen), and the fill collar (for refilling the Dewar). The
cryostatic cooling systems provide the following features: cooling of the detector to obtain stable
operating temperatures; high quality vacuum in the cryostat to avoid adsorption of contaminants
on the detector surface and provides thermal insulation; suppression of heat transfer between
cool inner parts and warm outer surface of the cryostat; mounting for the electrical contacts; and
isolation from external vibration to avoid system noise interference.\vspace{2 mm}

The Shielding involves an outer jacket of 10 mm carbon steel, an inner shield of 100-mm-thick
Lead and a graded liner of tin and copper. Such graded shielding minimizes the background
noise that has a very high probability of interference with measurement data. It is manufactured
for use with a detector in a vertical configuration.\vspace{2 mm}

\begin{table}
    \centering
    \begin{tabular}{l|l} \hline
        \textbf{Component} & \textbf{Specification} \\ \hline
        Detector model & GC3018 \\
        Relative efficiency & $\geq$ 30\% relative to NaI(Tl) \\
        Energy resolution & $\leq$ 1.8 keV at 1.33 MeV (Co-60) \\
        Depletion voltage & $+$ 2400 Vdc \\
        Recommended operating voltage & $+$ 2900 Vdc \\ \hline
    \end{tabular}
    \caption{Specifications of the HPGe Detector System}
\end{table}

%Adding 4 images in a 2x2 grid
\begin{figure}[h!]
    \centering
    \begin{tabular}{cc}
        \includegraphics[width=0.4\textwidth]{/Users/souvikpc/Desktop/Lab-Reports/3rd Sem/Experiments_with_HPGe_Detector /Figures/Apparatus/Schematic.png} &
        \includegraphics[width=0.4\textwidth]{/Users/souvikpc/Desktop/Lab-Reports/3rd Sem/Experiments_with_HPGe_Detector /Figures/Apparatus/ Detector unit.png} \\
        (a) Schematic Diagram of HPGe Detector & (b) HPGe Detector unit in CMRP\\
        \includegraphics[width=0.4\textwidth]{/Users/souvikpc/Desktop/Lab-Reports/3rd Sem/Experiments_with_HPGe_Detector /Figures/Apparatus/Upper view of Detector.png} &
        \includegraphics[width=0.4\textwidth]{/Users/souvikpc/Desktop/Lab-Reports/3rd Sem/Experiments_with_HPGe_Detector /Figures/Apparatus/Electronics part.png} \\
        (c) Upper view of Detector with graded shielding.  & (d) Electronics part of HPGe Detector
 \\
    \end{tabular}
    \caption{Images of the HPGe Detector System\cite{Lab}}
\end{figure}

\begin{table}
    \centering
    \begin{tabular}{l|l} 
        \hline
        \textbf{Component} & \textbf{Specification} \\ \hline
        Diameter of crystal & 61.7 mm \\
        Length of crystal & 40 mm \\
        Distance from crystal to front face & 4.9 mm \\
        Window material & 1.5 mm thick Aluminum \\
        Active volume & $\sim$ 113 cm$^{3}$ \\
        End cap & 3.00 inch diameter $\times$ 5.25 inch length \\\hline
    \end{tabular}
    \caption{Physical Characteristics of the HPGe Detector}
\end{table}

\subsection{Electronics components associated}
The HPGe detector system is complemented by a suite of electronic components
that facilitate signal processing, amplification, and data acquisition. Key components include:

\begin{itemize}
    \item \textbf{Pre-amplifier:} High-purity Ge detectors are usually fitted with a charge-sensitive 
    preamplifier, which acts as an interface between the detector crystal and the pulse-processing and 
    analysis electronics further along the gamma-spectrometric system. The preamplifier is often assembled as an integral part
    of the detector housing itself. It takes the charge produced from the detector (by the gamma
    radiation from the sample) and integrates and amplifies this to produce a step-function pulse. The
    amplitude of pulse is proportional to the total charge. The first stage usually includes a Field
    Effect Transistor (FET) circuit, which is located inside, or adjacent to the cryostat and is also
    cooled to reduce background-noise interference.
    \item \textbf{Amplifier:} The amplifier primarily takes the pulse
    signal from the preamplifier and considerably magnifies it. It also filters and shapes the incoming
    pulse to enhance the signal-to-noise ratio. This improves the resolution and shortens the response
    time to prevent overlap between pulses. Count rates for radionuclides in environmental samples
    are generally less than 100 counts per second, thus the amplifier needs to perform best in this
    range.
    \item \textbf{Analog-to-Digital Converter (ADC):} The analogue signal produced by the detector and shaped by the amplifier needs to be converted
    to a digital signal prior to registering in the MCA. This is undertaken using an analogue to digital
    converter (ADC), which effectively converts the analogue signal from the amplifier to a digital
    value.The pulses that emerge from the ADC are then registered in one of the channels of the
    multi-channel analyzer (MCA). The MCA is often hardwired into the computer system (via an
    electronic circuit inserted into the motherboard). The MCA provides the means by which the
    counts from the detector are stored according to the energy that produced them. It performs a
    number of tasks including collecting and sorting the input pulses, storing those data in a
    spectrum, providing a format to display the data on the computer screen, and performing some
    analysis of the data.
    \item \textbf{Data Acquisition System (DAQ):} Collects and stores the digital data for analysis in the software.
\end{itemize}

\section{Theory}
\subsection{Energy calibration of HPGe detector}
Energy calibration is the process of establishing a relationship between the channel
number in the multichannel analyzer (MCA) and the corresponding gamma-ray energy. This is
typically done using known gamma-ray sources with well-defined energies. The calibration
process involves the following steps:\vspace{2 mm}
\begin{itemize}
    \item \textbf{Selection of Calibration Sources:} Choose gamma-ray sources with known
    energies that cover the energy range of interest. Common calibration sources include
    Cs-137, Co-60, and Na-22.
    \item \textbf{Data Acquisition:} Place the calibration sources in front of the
    HPGe detector and acquire spectra for a sufficient duration to obtain clear peaks.
    \item \textbf{Peak Identification:} Identify the peaks in the acquired spectra
    corresponding to the known gamma-ray energies of the calibration sources.
    \item \textbf{Calibration Curve:} Plot the channel numbers of the identified peaks
    against their known energies. Fit a linear or polynomial function to the data points
    to establish the calibration curve.
    \item \textbf{Verification:} Verify the calibration by measuring additional known
    sources and checking if the measured energies match the expected values.
\end{itemize}

\subsection{Used radio active sources}
% make figures with 3 by 2 grid
\begin{figure}[H]
    \centering
    \begin{tabular}{ccc}
        \includegraphics[width=0.3\textwidth]{/Users/souvikpc/Desktop/Lab-Reports/3rd Sem/Experiments_with_HPGe_Detector /Figures/Radioisotopes/Cs-137_BM.pdf} &
        \includegraphics[width=0.3\textwidth]{/Users/souvikpc/Desktop/Lab-Reports/3rd Sem/Experiments_with_HPGe_Detector /Figures/Radioisotopes/Co-60_BM.pdf} &
        \includegraphics[width=0.3\textwidth]{/Users/souvikpc/Desktop/Lab-Reports/3rd Sem/Experiments_with_HPGe_Detector /Figures/Radioisotopes/Na-22_BP.pdf} \\
        (a) Cs-137 & (b) Co-60 & (c) Na-22 \\ 
        \includegraphics[width=0.3\textwidth]{/Users/souvikpc/Desktop/Lab-Reports/3rd Sem/Experiments_with_HPGe_Detector /Figures/Radioisotopes/Co-57_BP.pdf} &
        \includegraphics[width=0.3\textwidth]{/Users/souvikpc/Desktop/Lab-Reports/3rd Sem/Experiments_with_HPGe_Detector /Figures/Radioisotopes/Ba-133_BP.pdf} \\
        (d) Co-57 & (e) Ba-133  \\
    \end{tabular}
    \caption{Images of Radioactive Sources used in the Experiment\cite{LNHB_LARAWEB}}
\end{figure}

\subsection{Absolute efficiency calibration of HPGe detector}\
Absolute efficiency calibration involves determining the efficiency of the HPGe
detector for detecting gamma rays at different energies. This is essential for
quantifying the activity of radionuclides in samples. The calibration process includes
the following steps:\vspace{2 mm}
\begin{itemize}
    \item \textbf{Selection of Calibration Sources:} Use gamma-ray sources with known
    activities and energies.
    \item \textbf{Data Acquisition:} Place the calibration sources in front of the
    HPGe detector and acquire spectra for a known duration.
    \item \textbf{Peak Area Calculation:} Identify the peaks in the spectra and
    calculate the area under each peak, which corresponds to the number of counts
    detected for that energy.
    \item \textbf{Efficiency Calculation:} Calculate the absolute efficiency
    using the formula:
    \[ \epsilon(E) = \frac{N(E)}{A \cdot P(E) \cdot t} \]
    where \( \epsilon(E) \) is the absolute efficiency at energy \( E \), \( N(E) \)
    is the net peak area, \( A \) is the activity of the source, \( P(E) \) is the
    emission probability for the gamma ray at energy \( E \), and \( t \) is the counting time.
    \item \textbf{Efficiency Curve:} Plot the efficiency values against the corresponding
    energies and fit a suitable function to obtain the efficiency curve.
\end{itemize}

\subsection{Indentification of radionuclides and measurment of their activity concentration present in environmental sample}
To identify radionuclides and measure their activity concentration in environmental
samples using the calibrated HPGe detector, follow these steps:\vspace{2 mm}
\begin{itemize}
    \item \textbf{Sample Preparation:} Collect and prepare environmental samples
    (soil, water, etc.) ensuring they are in a suitable form for gamma spectroscopy.
    \item \textbf{Data Acquisition:} Place the prepared samples in front of the
    HPGe detector and acquire spectra for a sufficient duration to obtain clear peaks.
    \item \textbf{Peak Identification:} Analyze the acquired spectra to identify
    peaks corresponding to different radionuclides using the energy calibration curve.
\end{itemize}



\section{Observation \& Analysis}   
\subsection{Energy calibration}
For the purpose of energy calibration, standard gamma-emitting radionuclides—Cobalt-60, Cobalt-57, Ba-133, Na-22 and
Cesium-137—were employed. The gamma-ray spectra of each nuclide were recorded under identical conditions with an
acquisition time of 100 seconds. For each spectrum, the channel number corresponding to the centroid (peak position)
of the Gaussian-shaped photopeak was determined. The known gamma-ray energies associated with these radionuclides were then plotted against the
respective channel numbers to establish a calibration curve. A linear regression analysis was performed to derive the
calibration equation, which relates channel number to energy in keV. This calibration curve is essential for accurately determining the energies of unknown gamma-ray peaks in subsequent measurements.
%include a image
\begin{figure}[H]
    \centering
    \includegraphics[width=0.499\textwidth]{/Users/souvikpc/Desktop/Lab-Reports/3rd Sem/Experiments_with_HPGe_Detector /Figures/Observations/Energy calibration.png}
    \caption{Energy Calibration Plot of HPGe Detector}
\end{figure}

\subsection{Absolute efficiency calibration}
To achieve absolute efficiency calibration of the HPGe detector, standard gamma-emitting radionuclides Cobalt-60, Cobalt-57, Ba-133, Na-22 and Cesium-137—were utilized. The absolute efficiency of the detector was determined by measuring the count rates of these radionuclides at known activity levels.
Using given activity we can find the activity in the time experiment using decay formula:
\[ A = A_0 e^{-\lambda t} \]
where \( A \) is the activity at time \( t \), \( A_0 \) is the initial activity, and \( \lambda \) is the decay constant of the radionucl
%include a table
\begin{table}[H]
    \centering
    \begin{tabular}{ll} \
        \textbf{Radionuclide} & \textbf{Activity (kBq)} \\ \midrule
        Ba-133 & 120.557 \\
        Cs-137 & 102.8275\\
        Co-57 & 4.9983\\
        Co-60 & 98.0729\\
        Na-22 & 45.475\\ \bottomrule
    \end{tabular}
    \caption{Calculated Activities for Standard Radionuclides using decay formula}
\end{table}
The measured count rates were then compared to the expected count rates calculated from the known activities and the detector's geometry. A calibration curve was established to relate the measured count rates to the absolute efficiency of the detector at different energies.
%include a image
\begin{figure}[H]
    \centering
    \includegraphics[width=0.499\textwidth]{/Users/souvikpc/Desktop/Lab-Reports/3rd Sem/Experiments_with_HPGe_Detector /Figures/Observations/Efficiency calibration.PNG}
    \caption{Absolute Efficiency Calibration Plot of HPGe Detector}
\end{figure}

\subsection{Identification of radionuclides and measurement of their activity concentration present in environmental samples}
High-purity germanium (HPGe) detectors are widely employed for high-resolution gamma spectrometry due to their superior energy resolution. In essence, radionuclides present in the soil emit gamma photons at characteristic energies, which interact with the germanium crystal of the detector. These interactions generate electrical signals proportional to the energies of the incident photons. The signals are then amplified and transmitted to a Multi-Channel Analyser (MCA), where they are displayed as a spectrum representing emission counts versus photon energy. The accompanying software processes the spectral data, converting peak counts into specific activity values through appropriate calibration procedures.

For this experiment, two types of samples—tap water (100 mL) and soil (from the CMRP garden) (181 g)—were prepared and subjected to gamma-ray spectrometric analysis using identical measurement conditions. The spectra for each sample were recorded with a counting time of 1000 seconds.
%include image
\begin{figure}[H]
    \centering
    \includegraphics[width=0.499\textwidth]{/Users/souvikpc/Desktop/Lab-Reports/3rd Sem/Experiments_with_HPGe_Detector /Figures/Observations/Soil.PNG}
    \caption{In the soil sample we can see the presence of K-40 radionuclides.}
\end{figure}

\begin{figure}[H]
    \centering
    \includegraphics[width=0.499\textwidth]{/Users/souvikpc/Desktop/Lab-Reports/3rd Sem/Experiments_with_HPGe_Detector /Figures/Observations/Water.PNG}
    \caption{In the water sample we can see the presence of K-40 radionuclides.}
\end{figure}

\section{Applications in medical physics}
High Purity Germanium (HPGe) detectors play a crucial role in medical physics, particularly in the field of nuclear medicine and radiation therapy. Their high energy resolution allows for precise identification and quantification of gamma-emitting radionuclides used in diagnostic imaging and therapeutic applications. HPGe detectors are employed in the calibration of radiopharmaceuticals, ensuring accurate dosimetry for patient treatments. Additionally, they are used in quality control processes to verify the purity and activity of radioactive sources, contributing to patient safety and effective treatment outcomes. The ability of HPGe detectors to provide detailed spectral information enhances the accuracy of imaging techniques such as Single Photon Emission Computed Tomography (SPECT) and Positron Emission Tomography (PET), ultimately improving diagnostic capabilities in medical practice.

\section{Conclusion}
The experiment successfully demonstrated the energy calibration and absolute efficiency calibration of the HPGe detector using standard gamma-emitting radionuclides. The established calibration curves enable accurate identification and quantification of radionuclides in environmental samples. The analysis of soil and tap water samples revealed the presence of K-40 radionuclides, highlighting the detector's capability for environmental radioactivity assessment. Overall, the HPGe detector proved to be an effective tool for high-resolution gamma spectrometry in environmental studies.