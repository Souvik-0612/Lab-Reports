\section{Objective}
\section{Apparatus}
\section{Theory}
\section{Observation}
\begin{table}[H]
\centering
\begin{tabular}{cccc}
\hline
\textbf{Dose} & \textbf{A} & \textbf{B} & \textbf{Net Intensity} \\
\hline
Bg & 444 & 444 & 0 \\
0.5 & 15282472 & 13975510 & 14628547 \\
1 & 24422620 & 23555965 & 23988848.5 \\
1.5 & 29594792 & 29917852 & 29755878 \\
2 & 34702132 & 35937210 & 35319227 \\
3 & 45098778 & 41206552 & 43152221 \\
UA & 32726517 & 34434890 & 33580259.5 \\
UB & 42256628 & 39502916 & 40879328 \\
\hline
\end{tabular}
\caption{Dose-dependent measurements of OSL.}
\label{tab:dose_intensity}
\end{table}


\begin{figure}[H]
    \centering
    \includegraphics[width=0.7\textwidth]{/Users/souvikpc/Desktop/Lab-Reports/3rd Sem/OSL/Figures/Experimental data/chart-2.pdf}
    \caption{Dose vs Average Intensity graph for OSLD.}
\end{figure}

From the graph, we can see that the relationship between dose and intensity is linear. Using the linear fit equation, we can calculate the unknown doses.
Using the linear fit equation:
\[\boxed{\text{Dose} = \frac{I-1.16\times10^7}{1.11\times10^{7}}}
\]
Where I = Intensity. 

So for unknown A: the intensity is 33580259.5
\[\text{Dose} = \frac{33580259.5 - 1.16\times10^7}{1.11\times10^{7}} = \boxed{1.98} \text{ Gy}\]
Similarly, for unknown B: the intensity is 40879328
\[\text{Dose} = \frac{40879328 - 1.16\times10^7}{1.11\times10^{7}} = \boxed{2.64} \text{ Gy}\]
So the relative errors in the calculated doses are:
\section{Conclusion}