\section{Objective}
\begin{itemize}
    \item To study the characteristics of Optically stimulated luminescence material
($Al_2O_3$:C-Alumina)
    \item To calibrate the OSL research Reader in terms of absorbed dose and find out the
unknown dose.
\end{itemize}
\section{Apparatus}
\begin{itemize}
    \item Alumina discs
    \item OSL Reader
    \item Radiation generating equipment (LINAC)
    \item Blue light Bleaching machine
\end{itemize}
\section{Theory}
Luminescence dosimetry comprises two relative dosimetry techniques: 
thermoluminescence (TL) and optically stimulated luminescence (OSL). Of the two, TL dosimetry is older and better known, having been in commercial use since 1960s; OSL
dosimetry, in commercial use since 1990s, is less known than TL dosimetry but
has several features that give it an advantage over TL dosimetry. Both the TL and
OSL phenomena became known soon after the discovery of ionizing radiation during
1890s, but remained limited to academic interest until 1950s when their potential
for practical use in radiation dosimetry for measurement of absorbed dose and in
archaeology for dating of archaeological artifacts was established.
In many respects TL and OSL are similar: they both have the same theoretical
background, they both can be described as a process of stimulated phosphorescence
occurring in a previously irradiated crystalline mineral referred to as a phosphor, and
in both the stimulation results in emission of visible light proportional to the dose
absorbed in the phosphor. The major difference between the two techniques is in the
agent triggering acceleration of phosphorescence: in the TL process the stimulating
agent is heat while the OSL process is stimulated with visible light.

\subsection{Process of Optically Stimulated Luminescence}
The free electrons and holes released by energetic charged particles migrate
through the insulator in their respective energy bands and they either recombine or
become trapped in an electron or hole trap, respectively, somewhere in the insulator.
Two categories of trap are known:
\begin{itemize}
    \item First category is called a \textbf{storage trap} and its purpose is to trap free charge carriers
during irradiation and release them during subsequent stimulation either with heat
(TL process) or with light (OSL process).
    \item Second category acts as a \textbf{recombination center} or luminescence center where
a charge carrier released from a storage trap recombines with a trapped charge
carrier of opposite sign. The recombination energy is at least partially emitted in
the form of visible or ultraviolet light. The nature of the recombination process
depends strongly on the relative magnitudes of the capture cross section of the
unfilled storage traps and filled recombination centers
\end{itemize}

During irradiation of the insulator
the secondary charged particles lift electrons into the conduction band either (i) from
the valence band leaving a free hole in the valence band or (ii) from an empty hole
trap thereby filling the hole trap. The system may approach thermal equilibrium in
various ways:
\begin{enumerate}
    \item Two free charge carriers meet and recombine (electron-hole recombination);
recombination energy is converted into heat.
    \item Free charge carrier recombines with a charge carrier of opposite sign trapped at a
luminescence (recombination) center; recombination energy is emitted as optical
fluorescence.
    \item Free charge carrier becomes trapped at a storage trap, eventually resulting in
natural phosphorescence or accelerated phosphorescence called TL when the
accelerating agent is heat and OSL when the accelerating agent is visible or
ultraviolet light.
\end{enumerate}
\begin{figure}[H]
    \centering
    \includegraphics[width=0.7\textwidth]{/Users/souvikpc/Desktop/Lab-Reports/3rd Sem/OSL/Figures/Theory/Process.png}
    \caption{Two-step processes of optically stimulated luminescence
(OSL). Parts (a) and (c) illustrate step 1, parts (b) and (d) illustrate step 2. \cite{PodgorsakRadiationPhysics}}
    \label{Fig:OSL_Process}
\end{figure}
Figure \ref{Fig:OSL_Process} illustrates the two major stages of the OSL dosimetry processes.
Parts (a) and (c) show “Step 1: Irradiation” that consists of the following components:
\begin{enumerate}
    \item Interaction (PE, CE, TP, and NPP) of photon with an atom of insulator (phosphor)
and release of energetic charged particle (photoelectron, Compton electron, PP
electron and positron).
    \item ropagation of the energetic charged particle through insulator and creation of
electron-hole (e-h) pairs through ionization of atoms or ions of the insulator.
(Note: only one e-h creation is shown; however, each energetic charged particle
creates thousands of electron-hole pairs).
    \item If the electron of the e-h pair was supplied with kinetic energy $E_K$ exceeding the
energy gap $E_g$, the electron is lifted from the valence band into the conduction
band (Note: if $E_K < E_g$, then both the electron and hole remain in the valence
band, but they are bound together into a neutral entity called exciton, and migrate
through the valence band).
    \item Migration of free electron through conduction band of the insulator.
    \item Trapping of electron into an empty electron trap.
    \item Migration of hole of the e-h pair through valence band of the insulator.
    \item Trapping of hole into an empty hole trap.
\end{enumerate}
Parts (b) and (d) of Fig. \ref{Fig:OSL_Process} show “Step 2: Readout'' of the OSL processes.  
Part (b) is for $E_e < E_h$, indicating that the electron trap is a storage trap and the hole trap is a luminescence center; part (d) is for $E_e > E_h$, indicating that in this situation the electron trap acts as a luminescence center and the hole trap acts as a storage trap.  
Part (b) has the following components:

\begin{enumerate}
    \item Ejection of trapped electron from the electron storage trap into the conduction band as a result of  exposure to light (OSL) of the previously irradiated insulator (phosphor).
    \item Migration of free electron through the conduction band of the insulator.
    \item Recombination of free electron with a trapped hole at a luminescence center.
    \item Emission of visible or ultraviolet light (OSL).
\end{enumerate}

Part (d) of Fig. \ref{Fig:OSL_Process} has the following components:

\begin{enumerate}
    \item Ejection of trapped hole from the hole storage trap into the valence band as a result of  exposure to light (OSL) of the previously irradiated insulator.
    \item Migration of free hole through the valence band of the insulator.
    \item Recombination of free hole with a trapped electron at a luminescence center.
    \item Emission of visible or ultraviolet light (OSL).
\end{enumerate}
\subsection{Method of stimulation}
\begin{itemize}
    \item Continuous-Wave (CW) Stimulation: In this method, the OSL dosimeter is contin-
uously illuminated with light of constant intensity. CW-OSL remains the most commonly
used stimulation technique due to its simplicity, reliability, and ease of implementation.The emitted luminescence signal typically follows an exponentially decreasing trend with
time. The output intensity depends on the initial trap population, the stimulation power,
and the duration of exposure. For a fixed illumination time and constant stimulation
power, the integrated luminescent output is directly proportional to the initial number of
filled traps in the dosimetric material.
    \item Linearly Modulated (LM) Stimulation: In LM-OSL, the intensity of the stimulating
light increases linearly with time, starting from zero and reaching a maximum value as
the traps depopulate. The luminescence output exhibits distinct peaks corresponding to
different trap depths. This feature is particularly useful when analyzing materials with
multiple trapping levels, as it helps separate contributions from traps of different energies.
    \item Pulsed Stimulation: In pulsed OSL systems, short bursts of high-intensity light are used
to stimulate the dosimeter, while luminescence is measured between successive pulses.
Optical filters are employed to minimize contamination from scattered stimulation light;
however, since the intensity of the stimulating beam is several orders of magnitude higher
than that of the emitted luminescence, some leakage may still reach the detector. To
overcome this, the pulsed-OSL system is designed so that the detection electronics are
gated—i.e., the photomultiplier tube (PMT) output is recorded only during the dark
intervals between light pulses. During each pulse, electrons are excited from traps to
the conduction band. When the pulse ends, these electrons recombine at luminescent
centers, producing measurable light emission. This method significantly improves the
signal-to-noise ratio and reduces background interference, making it particularly effective
for low-dose measurements.
% Insert images 3 by 1 grid
\begin{figure}[H]
    \centering
    \begin{subfigure}{0.3\textwidth}
        \centering
        \includegraphics[width=\textwidth]{/Users/souvikpc/Desktop/Lab-Reports/3rd Sem/OSL/Figures/Light stimultion/CW.png}
        \caption{CW-OSL}
    \end{subfigure}
    \begin{subfigure}{0.3\textwidth}
        \centering
        \includegraphics[width=\textwidth]{/Users/souvikpc/Desktop/Lab-Reports/3rd Sem/OSL/Figures/Light stimultion/LM.png}
        \caption{LM-OSL}
    \end{subfigure}
    \begin{subfigure}{0.3\textwidth}
        \centering  
        \includegraphics[width=\textwidth]{/Users/souvikpc/Desktop/Lab-Reports/3rd Sem/OSL/Figures/Light stimultion/Pulsed.png}
        \caption{Pulsed-OSL}
    \end{subfigure}
    \caption{Different methods of light stimulation in OSL dosimetry.\cite{BotterJensenOSL}}
\end{figure}


\end{itemize}
\subsection{Bleaching Procedure}
After each readout, a certain fraction of trapped charge carriers may remain within the de-
fect centers of the OSL phosphor. These residual trapped electrons contribute to background
signals and can affect the accuracy of subsequent dose measurements if not properly removed.
Therefore, a bleaching process is carried out to eliminate the effects of any previous exposure
and to stabilize the electron traps for reuse of the phosphor.

Bleaching is performed using blue light illumination, which provides sufficient optical energy
to release the remaining trapped electrons without causing thermal or structural damage to
the phosphor material. In this procedure, all OSL discs are exposed to intense blue light for a
sufficient duration to ensure complete optical bleaching. Typically, the phosphors are subjected
to a blue light source of approximately 100 mW/cm2 for about 10 minutes, or alternatively, to
continuous blue light exposure for a total period of at least 30 minutes.

This optical bleaching effectively empties most of the shallow and intermediate traps,
thereby minimizing the residual signal and allowing the OSL phosphors to be reused for sub-
sequent irradiations with reliable and reproducible results.
\begin{figure}[H]
    \centering
    \includegraphics[width=0.7\textwidth]{/Users/souvikpc/Desktop/Lab-Reports/3rd Sem/OSL/Figures/Theory/Bleaching.png}
    \caption{Bleaching process of OSLD using blue light source.}
\end{figure}

\subsection{Types of OSLD}
The Al2O3 : C phosphor exhibits strong sensitivity to low doses of ionizing radiation. Its
main OSL emission peak occurs near 420 nm with a decay constant of approximately 35 ms,
while a faster emission component appears around 335 nm with a decay constant of less than
7 ns. Both of these luminescence bands are produced when trapped charge carriers in the
dosimeter are released by optical stimulation, typically using green light centered around
525 nm. Under continuous-wave (CW) OSL stimulation, maximum luminescence output is
obtained when the stimulation wavelength is close to 500 nm.
\begin{table}[ht]
\centering
\resizebox{\textwidth}{!}{
\begin{tabular}{l l c c c c c  c}
\hline
Material & Commercial Name & $\rho$ & $Z_{\text{eff}}$ & Glow Peak ($^\circ$C) & Emission (nm)  & Fading \\
\hline
Al$_2$O$_3$:C & nanoDot & 3.95 & 11.28 & $\sim$200 & $\sim$410 & 4\% in 3 months \\

BeO & Thermolox 995 & 2.85 & 7.21 & $\sim$210,330 & $\sim$335,390 & 5--10\% in 3 months \\

\hline
\end{tabular}
}
\end{table}
\subsection{OSL Research Reader}
PC controlled TL/OSL reader manufactured by Nucleonix systems is a compact integral unit,
designed primarily to meet the requirements of TL/OSL research community in R\&D labs and
universities who are engaged in luminescence studies of TL/OSL material. Data acquisition
is controlled by PC software. In OSL, optical stimulation by Blue and Green LED is also
controlled by PC software and electronic circuits and embedded code in the microcontroller.
System can be operated in TL or OSL modes required by the user.

The TL/OSL stimulation and detection chamber is a precisely fabricated, light-leakage-free
mechanical assembly housing a photon counting module with a detection filter basket. The
LED stimulation assemblies are positioned diagonally around the photon counting module
within a cylindrical enclosure. A dedicated OSL sample holder, integrated with a Kanthal
heater strip and a driver circuit, is also built into the chamber.
\begin{figure}
    \centering
    \includegraphics[width=0.7\textwidth]{/Users/souvikpc/Desktop/Lab-Reports/3rd Sem/OSL/Figures/Reader.png}
    \caption{Schematic diagram of OSL Research Reader.}
\end{figure}

The optical stimulation system consists of blue and green LED clusters, each LED having
a power rating of 3 W. Either the blue or green LED cluster (each containing two LEDs) can
be operated during stimulation. The LEDs are positioned diagonally at 180◦ with suitable lens
arrangements to provide uniform luminous intensity over the sample area.

The blue LED clusters, each with 3 W output and placed 180◦ opposite to each other, provide
stimulation with a peak emission wavelength of 465 nm and an emission band of 460–470 nm.
The green LED clusters have a similar configuration. Each LED assembly is equipped with a
long-pass stimulation filter of 420 nm cutoff and 12.5 mm diameter, which prevents scattered
light below 420 nm from entering the photomultiplier tube (PMT) directly. A suitable focusing
plano-concave lens is placed in front of each LED to focus the stimulation light onto the OSL
sample positioned in the planchet.

\subsection{Filter Baskets and Their Configurations}
For Optically Stimulated Luminescence (OSL) measurements, the detection system utilizes the
U-340 filter basket. This basket is designed to selectively transmit the luminescence signal
while effectively blocking the stimulating light, ensuring accurate photon counting from the
dosimeter.
The U-340 detection filter assembly provides an effective optical thickness of 7.5 mm, typi-
cally achieved by stacking three individual U-340 filters. These filters are arranged with O-rings
placed between each layer to ensure mechanical stability and prevent light leakage. The assem-
bly is secured using a chuck nut, which must be tightened sufficiently to hold the filters firmly
in position without applying excessive pressure that could damage the glass.
The U-340 filter has a peak transmission around 340 nm and an effective blocking range
for longer wavelengths, making it suitable for blue-light stimulation OSL measurements. This
configuration enhances the signal-to-noise ratio by allowing only the emitted luminescence to
reach the photomultiplier tube (PMT) while minimizing scattered stimulation light.

\section{Observation}
After irradiation of the OSLD with known doses, the OSL reader gives the intensity counts for each disc. The following table shows the dose-dependent measurements of OSL.
\begin{table}[H]
\centering
\begin{tabular}{cccc}
\hline
\textbf{Dose} & \textbf{A} & \textbf{B} & \textbf{Net Intensity} \\
\hline
Bg & 444 & 444 & 0 \\
0.5 & 15282472 & 13975510 & 14628547 \\
1 & 24422620 & 23555965 & 23988848.5 \\
1.5 & 29594792 & 29917852 & 29755878 \\
2 & 34702132 & 35937210 & 35319227 \\
3 & 45098778 & 41206552 & 43152221 \\
UA & 32726517 & 34434890 & 33580259.5 \\
UB & 42256628 & 39502916 & 40879328 \\
\hline
\end{tabular}
\caption{Dose-dependent measurements of OSL.}
\label{tab:dose_intensity}
\end{table}


\begin{figure}[H]
    \centering
    \includegraphics[width=0.7\textwidth]{/Users/souvikpc/Desktop/Lab-Reports/3rd Sem/OSL/Figures/Experimental data/chart-2.pdf}
    \caption{Dose vs Average Intensity graph for OSLD.}
\end{figure}

From the graph, we can see that the relationship between dose and intensity is linear. Using the linear fit equation, we can calculate the unknown doses.
Using the linear fit equation:
\[\boxed{\text{Dose} = \frac{I-1.16\times10^7}{1.11\times10^{7}}}
\]
Where I = Intensity. 

So for unknown A: the intensity is 33580259.5
\[\text{Dose} = \frac{33580259.5 - 1.16\times10^7}{1.11\times10^{7}} = \boxed{1.98} \text{ Gy}\]
Similarly, for unknown B: the intensity is 40879328
\[\text{Dose} = \frac{40879328 - 1.16\times10^7}{1.11\times10^{7}} = \boxed{2.64} \text{ Gy}\]
\subsubsection*{Error calculation:}
\begin{table}[H]
\centering
\begin{tabular}{cccc}
\hline
\textbf{Unknown} &\textbf{Calculated Dose (Gy)} & \textbf{Actual Dose (Gy)} & \textbf{Relative Error (\%)} \\
\hline
A & 1.98 & 1.78 & 11.24 \\
B & 2.64 & 2.54 & 3.93 \\ \hline
\end{tabular}
\end{table}
\section{Applications}
\begin{itemize}
    \item OSLDs are widely used in radiation therapy for patient dose monitoring and verification.
    \item They are employed in environmental radiation monitoring to assess exposure levels in various settings.
    \item OSLDs are utilized in personal dosimetry for occupational radiation workers to ensure safety and compliance with regulatory limits.
    \item They find applications in space missions for monitoring cosmic radiation exposure to astronauts.
    \item OSLDs are also used in archaeological dating and geological studies to determine the age of artifacts and sediments.
\end{itemize}
\section{Advantages of OSL over TLD}
\begin{itemize}
    \item OSLDs can be read multiple times without significant loss of signal, whereas TLDs can only be read once.
    \item OSLDs have a faster readout process compared to TLDs, allowing for quicker dose assessments.
    \item OSLDs exhibit higher sensitivity and a wider dynamic range than TLDs, making them suitable for low-dose measurements.
    \item The readout of OSLDs is non-destructive, preserving the dosimeter for future use, while TLDs require heating which destroys the stored information.
    \item OSLDs are less affected by environmental factors such as humidity and temperature compared to TLDs.
\end{itemize}
\section{Disadvantages of OSL}
\begin{itemize}
    \item OSLDs may require specialized equipment for readout, which can be costly.
    \item They may have a limited shelf life due to potential fading of the stored signal over time.
    \item OSLDs can be sensitive to light exposure, which may lead to unintended signal loss if not handled properly.
    \item The calibration of OSLDs can be more complex compared to TLDs, requiring careful consideration of the readout conditions.
\end{itemize}

\section{Conclusion}
From the experiment, we can conclude that the OSLD shows a linear relationship between dose and intensity. Using this linear relationship, we can calculate unknown doses with reasonable accuracy. The relative error for the calculated doses of unknown samples A and B are 11.24\% and 3.93\% respectively, indicating that the OSLD is a reliable dosimetry method for measuring absorbed doses in radiation therapy and other applications.