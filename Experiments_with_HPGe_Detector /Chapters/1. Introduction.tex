\section{Objective}
\begin{itemize}
    \item Energy calibration of HPGe detector
    \item Absolute efficiency calibration of HPGe detector
    \item Identification of radionuclides and measurement of their activity 
    concentration present in environmental soil sample and tap water sample.
\end{itemize}
\section{Apparatus}
\subsection{Introduction}
High Purity Germanium (HPGe) detectors are semiconductor detectors widely 
used in gamma spectroscopy due to their superior energy resolution compared to 
other detector types like NaI(Tl) scintillation detectors. They are particularly 
effective for identifying and quantifying gamma-emitting radionuclides in various
samples, including environmental samples.\vspace{2 mm}

Due to higher atomic number (Z=32), Ge has a much larger linear attenuation
coefficient for gamma rays than Si (Z=14). Which leads to a shorter mean free path
for gamma rays in Ge than in Si. This property makes Ge detectors more efficient
at detecting gamma rays, especially at higher energies.\vspace{2 mm}

The HPGe detectors must be operate at cryogenic temperatures (typically around 77 K,
the boiling point of liquid nitrogen) to reduce thermal noise and improve their
performance. This cooling is necessary because germanium has a relatively small bandgap,
and at room temperature, thermal energy can excite electrons across the bandgap,
leading to increased noise and reduced energy resolution.

\subsection{Principle of operation}
When a gamma ray interacts with the germanium crystal in the HPGe detector,
it can undergo several types of interactions, including the photoelectric effect,
Compton scattering, and pair production. These interactions result in the creation
of electron-hole pairs within the crystal. \textbf{The number of electron-hole pairs generated
is proportional to the energy of the incoming gamma ray}\vspace{2 mm}

An electric field is applied across the germanium crystal, causing the electrons
and holes to drift towards their respective electrodes. This movement of charge carriers
generates a current pulse, which is then amplified and processed by the detector's electronics.
The amplitude of the pulse is directly proportional to the energy of the incident gamma ray,
allowing for precise energy measurements.\vspace{2 mm}

\subsection{Construction and Cryostatic colling}
The HPGe detector consists of a high-purity germanium crystal, typically in the form
of a cylindrical or coaxial shape. The crystal is housed in a vacuum-sealed cryostat
to maintain the low temperatures required for operation. The cryostat is often cooled
using liquid nitrogen, which is circulated around the detector to keep it at the
desired temperature.\vspace{2 mm}

The main components of the system are the cooling rod (usually made of copper), the Dewar
container (for storage of liquid nitrogen), and the fill collar (for refilling the Dewar). The
cryostatic cooling systems provide the following features: cooling of the detector to obtain stable
operating temperatures; high quality vacuum in the cryostat to avoid adsorption of contaminants
on the detector surface and provides thermal insulation; suppression of heat transfer between
cool inner parts and warm outer surface of the cryostat; mounting for the electrical contacts; and
isolation from external vibration to avoid system noise interference.\vspace{2 mm}

The Shielding involves an outer jacket of 10 mm carbon steel, an inner shield of 100-mm-thick
Lead and a graded liner of tin and copper. Such graded shielding minimizes the background
noise that has a very high probability of interference with measurement data. It is manufactured
for use with a detector in a vertical configuration.\vspace{2 mm}

\begin{table}
    \centering
    \begin{tabular}{l|l} \hline
        \textbf{Component} & \textbf{Specification} \\ \hline
        Detector model & GC3018 \\
        Relative efficiency & $\geq$ 30\% relative to NaI(Tl) \\
        Energy resolution & $\leq$ 1.8 keV at 1.33 MeV (Co-60) \\
        Depletion voltage & $+$ 2400 Vdc \\
        Recommended operating voltage & $+$ 2900 Vdc \\ \hline
    \end{tabular}
    \caption{Specifications of the HPGe Detector System}
\end{table}

%Adding 4 images in a 2x2 grid
\begin{figure}[h!]
    \centering
    \begin{tabular}{cc}
        \includegraphics[width=0.4\textwidth]{/Users/souvikpc/Desktop/Lab-Reports/Experiments_with_HPGe_Detector /Figures/Apparatus/Schematic.png} &
        \includegraphics[width=0.4\textwidth]{/Users/souvikpc/Desktop/Lab-Reports/Experiments_with_HPGe_Detector /Figures/Apparatus/ Detector unit.png} \\
        (a) Schematic Diagram of HPGe Detector & (b) HPGe Detector unit in CMRP\\
        \includegraphics[width=0.4\textwidth]{/Users/souvikpc/Desktop/Lab-Reports/Experiments_with_HPGe_Detector /Figures/Apparatus/Upper view of Detector.png} &
        \includegraphics[width=0.4\textwidth]{/Users/souvikpc/Desktop/Lab-Reports/Experiments_with_HPGe_Detector /Figures/Apparatus/Electronics part.png} \\
        (c) Upper view of Detector with graded shielding.  & (d) Electronics part of HPGe Detector
 \\
    \end{tabular}
    \caption{Images of the HPGe Detector System}
\end{figure}

\begin{table}
    \centering
    \begin{tabular}{l|l} 
        \hline
        \textbf{Component} & \textbf{Specification} \\ \hline
        Diameter of crystal & 61.7 mm \\
        Length of crystal & 40 mm \\
        Distance from crystal to front face & 4.9 mm \\
        Window material & 1.5 mm thick Aluminum \\
        Active volume & $\sim$ 113 cm$^{3}$ \\
        End cap & 3.00 inch diameter $\times$ 5.25 inch length \\\hline
    \end{tabular}
    \caption{Physical Characteristics of the HPGe Detector System}
\end{table}

\subsection{Electronics components associated}
The HPGe detector system is complemented by a suite of electronic components
that facilitate signal processing, amplification, and data acquisition. Key components include:

\begin{itemize}
    \item \textbf{Pre-amplifier:} High-purity Ge detectors are usually fitted with a charge-sensitive 
    preamplifier, which acts as an interface between the detector crystal and the pulse-processing and 
    analysis electronics further along the gamma-spectrometric system. The preamplifier is often assembled as an integral part
    of the detector housing itself. It takes the charge produced from the detector (by the gamma
    radiation from the sample) and integrates and amplifies this to produce a step-function pulse. The
    amplitude of pulse is proportional to the total charge. The first stage usually includes a Field
    Effect Transistor (FET) circuit, which is located inside, or adjacent to the cryostat and is also
    cooled to reduce background-noise interference.
    \item \textbf{Amplifier:} The amplifier primarily takes the pulse
    signal from the preamplifier and considerably magnifies it. It also filters and shapes the incoming
    pulse to enhance the signal-to-noise ratio. This improves the resolution and shortens the response
    time to prevent overlap between pulses. Count rates for radionuclides in environmental samples
    are generally less than 100 counts per second, thus the amplifier needs to perform best in this
    range.
    \item \textbf{Analog-to-Digital Converter (ADC):} The analogue signal produced by the detector and shaped by the amplifier needs to be converted
    to a digital signal prior to registering in the MCA. This is undertaken using an analogue to digital
    converter (ADC), which effectively converts the analogue signal from the amplifier to a digital
    value.The pulses that emerge from the ADC are then registered in one of the channels of the
    multi-channel analyzer (MCA). The MCA is often hardwired into the computer system (via an
    electronic circuit inserted into the motherboard). The MCA provides the means by which the
    counts from the detector are stored according to the energy that produced them. It performs a
    number of tasks including collecting and sorting the input pulses, storing those data in a
    spectrum, providing a format to display the data on the computer screen, and performing some
    analysis of the data.
    \item \textbf{Data Acquisition System (DAQ):} Collects and stores the digital data for analysis in the software.
\end{itemize}



\section{Theory}

\section{Observation}
\section{Conclusion}