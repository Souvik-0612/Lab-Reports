\section{Objective}
To measure the Electron Beam output of a Medical Linear Accelerator.
\section{Apparatus}

\subsection*{BEAMSCAN Water Phantom System}

The BEAMSCAN system (PTW, Freiburg) is a modern, fully motorized
three-dimensional water phantom designed for beam data acquisition in
radiotherapy. It is widely used for commissioning, quality assurance, and
beam characterization of linear accelerators.

\begin{itemize}
  \item \textbf{Purpose:} High-precision scanning of photon and electron beams for dosimetric measurements.
  \item \textbf{Design:} Large-volume water tank with three orthogonal motorized axes (X, Y, Z) for detector positioning.
  \item \textbf{Automation:} Fully computer-controlled scanning with high reproducibility and sub-millimeter positioning accuracy.
  \item \textbf{Detectors:} Compatible with ionization chambers, diodes, and other field detectors for depth-dose and profile measurements.
  \item \textbf{Applications:} Beam commissioning, reference dosimetry, treatment planning system (TPS) data input, and periodic QA of radiotherapy machines.
  \item \textbf{Advantages:} High mechanical stability, waterproof detector holders, automated setup, and integration with dedicated software for data analysis.
\end{itemize}

\section{Theory}

\section{Observation}
\section{Observation in Acharya Harihar Post Graduate Institute of Cancer}

\subsection{Tabulation for 6MeV electron}
\begin{table}[H]
    \centering
    \begin{tabular}{ccccc}
    \toprule
        Bias Voltage  &  $M_{Q1}$   & $M_{Q2}$   & $M_{Q3}$  & Average ($M_{Qunc}$)\\ \midrule
      +300 V & 1.831 nC & 1.834 nC & 1.836 nC & 1.8336 nC \\
      +150 V & 1.821 nC& 1.824 nC & 1.829 nC& 1.8246 nC\\
      -300 V & -1.847 nC& -1.851 nC& -1.853 nC & -1.8503 nC\\
    \bottomrule
    \end{tabular}
    \caption{Tabulation for 6 MeV}
\end{table}

\subsubsection*{Correction for Temperature, Pressure, and Humidity}

\begin{equation}
    k_{T,P} = \bigg(\frac{273.15+T}{273.15+T_0}\bigg)\bigg(\frac{P_0}{P}\bigg)
    \label{eqn:K_{T,P}}
\end{equation}
where $T = 21.2 ^o$C, $T_0 = 20^o$C, $P_o = 101.3 $ kPa and $P = 100.3$ kPa.\


So, After putting in the value, $\boxed{k_{TP} = 1.014} $.


\subsubsection*{Correction for Ion Recombination/ Saturation:}

\begin{equation}
    k_s = a_0 + a_1\left(\frac{M_1}{M_2}\right)+a_3\left(\frac{M_1}{M_2}\right)^{2}
\end{equation}

where $a_0 = 2.79977$, $a_1 = -4.50337$, $a_2 = 2.70513$ for a voltage ratio of 2, and the values for $M_{1}$ and $M_2$ are the averages of the meter readings for +300 volts and +150 volts, respectively. So, $M_{1} = 1.8336$ and $M_{2} = 1.8246$.


After calculating , $\boxed{k_{s} = 1.00606}$.

\subsubsection*{Polarity Correction:}
\begin{equation}
    K_{pol} = \frac{|M_{+}|+ |M_{-}|}{2|M|}
\end{equation}

Putting the values $|M_{+}| = |M|$ and $|M_{-}|$ the $K_{Pol}$ calculation, we get $\boxed{K_{Pol}=1.005}$.

\subsubsection*{Corrected meter reading:}
\begin{align*}
    M_Q &= M' \times k_{pol} \times k_{sat} \times k_{TP}\\
        &= 1.8336 \times  1.005 \times 1.006 \times 1.014\\
        &= \boxed{ 1.8797 \text{ nC}}
\end{align*}

\subsubsection*{Absorbed does to water at $Z_{ref} = 1.4$ cm depth:}
Given, $k_{Q, Q_0} = 0.919$ and $N_{D, w} = 5.734 \times 10^8$ Gy/C
\begin{align*}
    D'_{w, Q} &= M_Q \times N_{D,w} \times k_{Q, Q_0}  \\
              &= 1.8797 \text{ nC} \times 5.734 \times 10^8\text{ Gy/C} \times 0.919 \\
              &= \boxed{0.99055 \text{ cGy}}
\end{align*}
\subsubsection*{Dose at the depth of dose maxima, $100$ cm SSD set up:}
PDD at $Z_{ref} = 1.4$ cm for a 10 cm $\times$ 10 cm field size for a 6 MeV beam is \textit{99.166} \%.

Absorbed dose rate calibration at $Z_{max}$ 
\begin{align*}
    D_{w, Q} &= \frac{0.99055 \times 100}{99.166} \\
             &= \boxed{0.9988 \text{ cGy/MU}} 
\end{align*}

\subsubsection*{Error calculation}
\begin{itemize}
  \item Output measured: \(0.9988 \, \text{cGy/MU} \)
  \item Standard output: \( 1.0000 \, \text{cGy/MU} \)
\end{itemize}



\begin{align*}
    \text{Error (\%)} &= \left( \frac{\text{Measured} - \text{Standard}}{\text{Standard}} \right) \times 100\\
            &= \left| \frac{0.9988 - 1.0000}{1.0000} \right| \times 100\\
            &= 0.12\% \text{ (Tolerance = 2\%) }
\end{align*}

\subsection{Tabulation for 12 MeV electron}
\begin{table}[H]
    \centering
    \begin{tabular}{ccccc}
    \toprule
        Bias Voltage  &  $M_{Q1}$   & $M_{Q2}$   & $M_{Q3}$  & Average ($M_{Qunc}$)\\ \midrule
      +300 V & 1.865 nC & 1.869 nC & 1.870 nC & 1.868 nC \\
      +150 V & 1.851 nC& 1.856 nC & 1.86 nC& 1.8556 nC\\
      -300 V & -1.869 nC& -1.872 nC& -1.874 nC & -1.8716 nC\\
    \bottomrule
    \end{tabular}
    \caption{Tabulation for 12 MeV}
\end{table}

\subsubsection*{Correction for Temperature, Pressure, and Humidity}

\begin{equation}
    k_{T,P} = \bigg(\frac{273.15+T}{273.15+T_0}\bigg)\bigg(\frac{P_0}{P}\bigg)
    \label{eqn:K_{T,P}}
\end{equation}
where $T = 21.2 ^o$C, $T_0 = 20^o$C, $P_o = 101.3 $ kPa and $P = 100.3$ kPa.\


So, After putting in the value, $\boxed{k_{TP} = 1.014} $.


\subsubsection*{Correction for Ion Recombination/ Saturation:}

\begin{equation}
    k_s = a_0 + a_1\left(\frac{M_1}{M_2}\right)+a_3\left(\frac{M_1}{M_2}\right)^{2}
\end{equation}

where $a_0 = 2.79977$, $a_1 = -4.50337$, $a_2 = 2.70513$ for a voltage ratio of 2, and the values for $M_{1}$ and $M_2$ are the averages of the meter readings for +300 volts and +150 volts, respectively. So, $M_{1} = 1.868$ and $M_{2} = 1.8556$.


After calculating , $\boxed{k_{s} = 1.0077}$.

\subsubsection*{Polarity Correction:}
\begin{equation}
    K_{pol} = \frac{|M_{+}|+ |M_{-}|}{2|M|}
\end{equation}

Putting the values $|M_{+}| = |M|$ and $|M_{-}|$ the $K_{Pol}$ calculation, we get $\boxed{K_{Pol}=1.001}$.

\subsubsection*{Corrected meter reading:}
\begin{align*}
    M_Q &= M' \times k_{pol} \times k_{sat} \times k_{TP}\\
        &= 1.868 \times  1.001 \times 1.007 \times 1.014\\
        &= \boxed{ 1.9093 \text{ nC}}
\end{align*}

\subsubsection*{Absorbed does to water at $Z_{ref} = 2.7$ cm depth:}
Given, $k_{Q, Q_0} = 0.919$ and $N_{D, w} = 5.734 \times 10^8$ Gy/C
\begin{align*}
    D'_{w, Q} &= M_Q \times N_{D,w} \times k_{Q, Q_0}  \\
              &= 1.9093 \text{ nC} \times 5.734 \times 10^8\text{ Gy/C} \times 0.919 \\
              &= \boxed{1.0061 \text{ cGy}}
\end{align*}
\subsubsection*{Dose at the depth of dose maxima, $100$ cm SSD set up:}
PDD at $Z_{ref} = 2.7$ cm for a 10 cm $\times$ 10 cm field size for a 6 MeV beam is \textit{98.963} \%.

Absorbed dose rate calibration at $Z_{max}$ 
\begin{align*}
    D_{w, Q} &= \frac{1.0061 \times 100}{98.963} \\
             &= \boxed{1.0166 \text{ cGy/MU}} 
\end{align*}

\subsubsection*{Error calculation}
\begin{itemize}
  \item Output measured: \(1.0166 \, \text{cGy/MU} \)
  \item Standard output: \( 1.0000 \, \text{cGy/MU} \)
\end{itemize}



\begin{align*}
    \text{Error (\%)} &= \left( \frac{\text{Measured} - \text{Standard}}{\text{Standard}} \right) \times 100\\
            &= \left| \frac{1.0166 - 1.0000}{1.0000} \right| \times 100\\
            &= 1.66\% \text{ (Tolerance = 2\%) }
\end{align*}


\section{Observation in AIIMS}

\subsection{Tabulation for 6 MeV electron beam}
\begin{table}[H]
    \centering
    \begin{tabular}{ccccc}
    \toprule
        Bias Voltage  &  $M_{Q1}$   & $M_{Q2}$   & $M_{Q3}$  & Average ($M_{Qunc}$)\\ \midrule
      +300 V & 683 pC & 682 pC & 681.5 pC & 682.17 pC \\
      +150 V & 680 pC& 679.5 pC & 680.5 pC& 680 pC\\
      -300 V & -692.5 pC& -692.5 pC& -691.5 pC & -692.17 pC\\
    \bottomrule
    \end{tabular}
    \caption{Tabulation for 6 MeV}
\end{table}

\subsubsection*{Correction for Temperature, Pressure, and Humidity}

\begin{equation}
    k_{T,P} = \bigg(\frac{273.15+T}{273.15+T_0}\bigg)\bigg(\frac{P_0}{P}\bigg)
    \label{eqn:K_{T,P}}
\end{equation}
where $T = 23 ^o$C, $T_0 = 20^o$C, $P_o = 101.3 $ kPa and $P = 101.1$ kPa.\


So, After putting in the value, $\boxed{k_{TP} = 1.0122} $.


\subsubsection*{Correction for Ion Recombination/ Saturation:}

\begin{equation}
    k_s = a_0 + a_1\left(\frac{M_1}{M_2}\right)+a_3\left(\frac{M_1}{M_2}\right)^{2}
\end{equation}

where $a_0 = 2.79977$, $a_1 = -4.50337$, $a_2 = 2.70513$ for a voltage ratio of 2, and the values for $M_{1}$ and $M_2$ are the averages of the meter readings for +300 volts and +150 volts, respectively. So, $M_{1} = 682.17$ and $M_{2} = 680$.


After calculating , $\boxed{k_{s} = 1.0044}$.

\subsubsection*{Polarity Correction:}
\begin{equation}
    K_{pol} = \frac{|M_{+}|+ |M_{-}|}{2|M|}
\end{equation}

Putting the values $|M_{+}| = |M|$ and $|M_{-}|$ the $K_{Pol}$ calculation, we get $\boxed{K_{Pol}=1.0073}$.

\subsubsection*{Corrected meter reading:}
\begin{align*}
    M_Q &= M' \times k_{pol} \times k_{sat} \times k_{TP}\\
        &= 682.17 \times  1.0073 \times 1.0044 \times 1.0122\\
        &= \boxed{ 698.593\text{ pC}}
\end{align*}

\subsubsection*{Absorbed does to water at $Z_{ref} = 1.4$ cm depth:}
Given, $k_{Q, Q_0} = 0.939$ and $N_{D, w} = 150.7$ cGy/nC
\begin{align*}
    D'_{w, Q} &= M_Q \times N_{D,w} \times k_{Q, Q_0}  \\
              &= 698.593 \text{ pC} \times 150.7\text{ cGy/nC} \times 0.939 \\
              &= \boxed{98.856 \text{ cGy}}
\end{align*}
\subsubsection*{Dose at the depth of dose maxima, $100$ cm SSD set up:}
PDD at $Z_{ref} = 2.7$ cm for a 10 cm $\times$ 10 cm field size for a 6 MeV beam is \textit{98.963} \%.

Absorbed dose rate calibration at $Z_{max}$ 
\begin{align*}
    D_{w, Q} &= \frac{98.856}{98.963} \\
             &= \boxed{1.0166 \text{ cGy/MU}} 
\end{align*}

\subsubsection*{Error calculation}
\begin{itemize}
  \item Output measured: \(1.0166 \, \text{cGy/MU} \)
  \item Standard output: \( 1.0000 \, \text{cGy/MU} \)
\end{itemize}



\begin{align*}
    \text{Error (\%)} &= \left( \frac{\text{Measured} - \text{Standard}}{\text{Standard}} \right) \times 100\\
            &= \left| \frac{1.0166 - 1.0000}{1.0000} \right| \times 100\\
            &= 1.66\% \text{ (Tolerance = 2\%) }
\end{align*}

\subsection{Tabulation for 12 MeV electron}
\begin{table}[H]
    \centering
    \begin{tabular}{ccccc}
    \toprule
        Bias Voltage  &  $M_{Q1}$   & $M_{Q2}$   & $M_{Q3}$  & Average ($M_{Qunc}$)\\ \midrule
      +300 V & 718.5 pC & 717.5 pC & 718 pC & 718 pC \\
      +150 V & 715.5 pC& 715.5 pC & 714 pC& 715 pC\\
      -300 V & -723.5 pC& -723 pC& -722.5 pC & -723 pC\\
    \bottomrule
    \end{tabular}
    \caption{Tabulation for 12 MeV}
\end{table}

\subsubsection*{Correction for Temperature, Pressure, and Humidity}

\begin{equation}
    k_{T,P} = \bigg(\frac{273.15+T}{273.15+T_0}\bigg)\bigg(\frac{P_0}{P}\bigg)
    \label{eqn:K_{T,P}}
\end{equation}
where $T = 23 ^o$C, $T_0 = 20^o$C, $P_o = 101.3 $ kPa and $P = 101.1$ kPa.\


So, After putting in the value, $\boxed{k_{TP} = 1.0122} $.


\subsubsection*{Correction for Ion Recombination/ Saturation:}

\begin{equation}
    k_s = a_0 + a_1\left(\frac{M_1}{M_2}\right)+a_3\left(\frac{M_1}{M_2}\right)^{2}
\end{equation}

where $a_0 = 2.79977$, $a_1 = -4.50337$, $a_2 = 2.70513$ for a voltage ratio of 2, and the values for $M_{1}$ and $M_2$ are the averages of the meter readings for +300 volts and +150 volts, respectively. So, $M_{1} = 718$ and $M_{2} = 715$.


After calculating , $\boxed{k_{s} = 1.00538}$.

\subsubsection*{Polarity Correction:}
\begin{equation}
    K_{pol} = \frac{|M_{+}|+ |M_{-}|}{2|M|}
\end{equation}

Putting the values $|M_{+}| = |M|$ and $|M_{-}|$ the $K_{Pol}$ calculation, we get $\boxed{K_{Pol}=1.00348}$.

\subsubsection*{Corrected meter reading:}
\begin{align*}
    M_Q &= M' \times k_{pol} \times k_{sat} \times k_{TP}\\
        &= 718 \times  1.00348 \times 1.00538 \times 1.0122\\
        &= \boxed{ 733.212\text{ pC}}
\end{align*}

\subsubsection*{Absorbed does to water at $Z_{ref} = 2.7$ cm depth:}
Given, $k_{Q, Q_0} = 0.919$ and $N_{D, w} = 5.734 \times 10^8$ Gy/C
\begin{align*}
    D'_{w, Q} &= M_Q \times N_{D,w} \times k_{Q, Q_0}  \\
              &= 733.212 \text{ nC} \times 5.734 \times 10^8\text{ Gy/C} \times 0.919 \\
              &= \boxed{1.0061 \text{ cGy}}
\end{align*}
\subsubsection*{Dose at the depth of dose maxima, $100$ cm SSD set up:}
PDD at $Z_{ref} = 2.7$ cm for a 10 cm $\times$ 10 cm field size for a 6 MeV beam is \textit{98.963} \%.

Absorbed dose rate calibration at $Z_{max}$ 
\begin{align*}
    D_{w, Q} &= \frac{1.0061 \times 100}{98.963} \\
             &= \boxed{1.0166 \text{ cGy/MU}} 
\end{align*}

\subsubsection*{Error calculation}
\begin{itemize}
  \item Output measured: \(1.0166 \, \text{cGy/MU} \)
  \item Standard output: \( 1.0000 \, \text{cGy/MU} \)
\end{itemize}



\begin{align*}
    \text{Error (\%)} &= \left( \frac{\text{Measured} - \text{Standard}}{\text{Standard}} \right) \times 100\\
            &= \left| \frac{1.0166 - 1.0000}{1.0000} \right| \times 100\\
            &= 1.66\% \text{ (Tolerance = 2\%) }
\end{align*}


\section{Conclusion}